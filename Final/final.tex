% QUESTO E' IL MODELLO DI DOCUMENTO FINALE DA PRESENTARE ALLA PROVA D'ESAME DI
% "SISTEMI INFORMATIVI" A.A. 2014/2015. E' BASATO SUL MODELLO ADOTTATO DA
% SPRINGER PER GLI ATTI DELLE PROPRIE CONFERENZE, COME SPIEGATO NEL SEGUITO. 

\documentclass{llncs}

% Si utilizzi il pacchetto babel se si scrive in italiano
\usepackage[italian]{babel}
\usepackage[applemac]{inputenc}
\usepackage{algorithmicx}
\usepackage{algpseudocode}
\usepackage{geometry}
\usepackage{graphicx}
\usepackage{caption}
\usepackage{hyperref}
\usepackage{subcaption}
\usepackage{mathtools}
\usepackage{float}

% \usepackage[caption=false]{subfig}

\begin{document}

\title{Documento finale per la prova d'esame di Sistemi Informativi
  a.a. 2014/2015} 
\author{
	Catalin Copil\\\email{catalin.copil@studenti.unipd.it}\\ 
	Mattia de Stefani\\\email{mattia.destefani@studenti.unipd.it}\\ 
	Giulio Lovisotto\\\email{giulio.lovisotto@studenti.unipd.it}} 
\institute{}

\date{\today}
% Just remember to make sure that the TOTAL number of authors
% is the number that will appear on the first page PLUS the
% number that will appear in the \additionalauthors section.

\maketitle
\begin{abstract}
Il presente documento descrive le attivit\`a di laboratorio svolte dagli autori (gruppo 12) durante il corso di Sistemi Informativi, tenuto dal professor M. Melucci presso l'Universit\`a di Padova, anno accademico 2014/2015. 
L'obiettivo delle tecniche utilizzate \`e quello di massimizzare l'efficacia della funzione di reperimento nella collezione di articoli scientifici CACM. All'interno di questo documento vengono riportate informazioni sulle scelte implementative dei metodi di Information Retrieval trattati. I componenti del gruppo hanno deciso di integrare nel reperimento le tecniche proposte scegliendo di partire da \textsc{bm25} come riferimento. In particolare, sono stati integrati gli algoritmi di \textsc{pagerank}, \textsc{relevance feedback}, \textsc{lsa}, \textsc{hits}. Inoltre ai fini dell'ottimizzazione \`e stato utilizzata un Evolution Strategy. Il documento comprende un'analisi ragionata dei risultati ottenuti, e termina con alcune conclusioni generali sul lavoro svolto.

%Questo documento illustra il perscorso effettuato per arrivare ad ottenere una buona conoscenza del ramo dell'Information Retrieval.
%Partendo dalle conoscenze di base ad ogni esercitazione è stato aggiunto un tassello, il quale non è altro che un'attributo in più da considerare per il calcolo della precisione dei file ritenuti da noi importanti.
%All'interno del documento saranno spiegati i diversi progetti effettuati. Come siamo passati da un'indicizzazione manuale ad una automatica (laboratorio 2), l'algoritmo di reperimento implementato per effettuare il ranking dei documenti presenti nella collezione per ogni richiesta, query (laboratorio 3). L'introduzione dei giudizi di rilevanza e la loro influenza sul risultato elaborato dall'algoritmo di reperimento scelto (laboratorio 4). Il calcolo effettuato del pagerank di ogni documento e come quest'ultimo in fluisca sul ranking (laboratorio 5). L'analisi della relazione tra una collezione di documenti e i termini contenuti in essi tramite l'utilizzo della tecnica \textbf{LSA} (laboratorio 6 ). Concludento con l'introduzione e lo studio di \textbf{HITS} (Hyperlink-Induced Topic Search): algoritmo di analisi dei link (laboratorio 7).  
	
	
\end{abstract}

\section{Introduzione}
\label{sec:introduzione}

http://www.guit.sssup.it/latex/suggerimenti.php

Introdurre brevemente lo scopo del corso. Che collezione usiamo (http://ir.dcs.gla.ac.uk/resources/test\_collections/cacm/). No stop words, no stemming (gia fatto). 
Abbiamo a disposiz il grafo delle citazioni quindi useremo \textsc{pagerank} e Hits (link analysis), spiegare brevemente cosa sono questi. Spiegare LSA. Spiegare ES. Related works?  citt\`e
\subsection{Organizzazione}
In sezione .. spiegeremo .., in sezione .. spiegheremo .. etc.

%Quando si scrive l'introduzione, si deve tenere presente che il lettore atteso
%sar\`a a conoscenza degli elementi di base. Si dovranno quindi introdurre i
%concetti non normalmente trattati in un corso di base in Information Retrieval e
%che sono invece utilizzati nel proprio documento.
%
%Il resto del documento sar\`a scritto con i criteri seguenti:
%\begin{itemize}
%\item esaustivit\`a
%\item precisione
%\item chiarezza
%\item correttezza
%\item sintesi
%\end{itemize}


\section{Metodologia}
\label{sec:metodologia}

Spiegare notazione grafo, parametri, usiamo 1 simbolo per indicare gli score. Quando scriviamo una sezione qui sotto 2.x, se ci serve della notazione poi andiamo a metterla qui. 
Fare dei paragrafi per BM25, PageRank, Hits, LSA, ES.

%In questo paragrafo, si illustreranno i metodi sviluppati e sperimentati con le
%attivit\`a di laboratorio. Le notazioni e tutti gli aspetti non banali dovranno
%essere spiegati. Naturalmente, la notazione di un paragrafo non dovr\`a essere
%reintrodotta nei paragrafi successivi, di conseguenza, la notazione non dovr\`a
%essere ambigua.

\subsection{Approccio}
\label{sec:approccio}

Vogliamo max la map. Usando treceval spiegare cos'e'. Spiegare brevemente il lavoro, molto veloce dire che abbiamo lavorato insieme dopo la prima sessione in lab. Usato python, con \texttt{numpy} per matrici, matplotlib per plottare, networkx per grafi, git per versionamento, latex per i report.

%Abbiamo deciso di utilizzare un approccio probabilistico poichè documentandoci siamo arrivati alla conclusione che quello vettoriale è ormai superato, e vi sono diverse implementazioni di modelli probabilistici (attualmente studiati) per diversi software le cui prestazioni sono elevate e ottimali.
%Inoltre per via delle nostre conoscenze abbiamo voluto utilizzare il linguaggio Python per dare corpo ai progetti. Esso ha un ottimo utilizzo di vettori, matrici e dizionari ottimali per i nostri scopi, al contrario di altri linguaggi molto spesso ci permette di scrivere in una riga metodi o passaggi che altrimenti avrebbero preso più spazio. Inoltre vi sono diverse librerie che implementano molti metodi utili per tale corso e per garantire buone performance al codice, così da rendere l'esecuzione veloce e snella e avere un tempo di elaborazione piccolo quindi poter permettere a noi utenti diverse prove fino al raggiungimento del risultato ottimale ricercato.
%Per ogni esercitazioni è stato utilizzato un approccio di gruoppo, cioè dataci le direttive del lavoro da svolgere ogni membro del gruppo rispolverava la teoria e pensava a come rispondere correttamente alla consegna. Al termine di questo primo step vi era la fase di scambo delle opinioni e conoscenze arrivando così a come doveva essere fatto il lavoro finale. Al dato giorno e ora ogni membro si trovava per la scrittura del codice e del report dell'esercitazione.
%Ci siamo sempre documentati prima di scrivere il codice cercando di utilizzare metodi presenti in librerie opportune così da avere permonca ottimali e poter effettuare più prove. Ogni metodo utilizzato è stato dapprima studiato per capirne il funzionamento e l'utilizzo per non avere dati rappresentati in una forma a noi non utile e non avere dati in più, che sarebbe stata solo memoria sprecata. 
%Diverse volte, invece, abbiamo scritto noi l'intera funzione per il calcolo di una particolare "cosa", ad esempio nel laboratorio  abbiamo deciso di implementare il metodo \textsc{bm25} (spiegato successivamente), studiata la teoria e le formule da applicare abbiamo scritto l'algoritmo che segue passo a passo il metodo citato ed effettuato diversi test per verificarne il corretto funzionamento.

\subsection{Indicizzazione} \label{sec:metodi-di-indic}

Qui va il contenuto del laboratorio n. 2.

\subsection{Reperimento}
\label{sec:metodi-di-reper}

Qui va il contenuto del laboratorio n. 3. Esso rappresenta la \textit{baseline}.

\subsection{Relevance Feedback}
\label{sec:relevance-feedback}

Qui va il contenuto del laboratorio n. 4.

\subsection{PageRank}
\label{sec:pagerank}

Qui va il contenuto del laboratorio n. 5.

\subsection{Latent Semantic Analysis}
\label{sec:lsa}

Qui va il contenuto del laboratorio n. 6.

\subsection{Hyper-linked Induced Topic Search}
\label{sec:hits}

Qui va il contenuto del laboratorio n. 7.

\subsection{Evolution Strategy}
\label{sec:es}

Per ottimizzare gli algoritmi di reperimento abbiamo scelto di utilizzare un Evolution Strategy~\cite{back1996evolutionary} (ES) che e' una tecnica di ottimizzazione basata sui principi che regolano l'evoluzione. Tecniche di questo tipo sono piu' robuste rispetto ai metodi di ricerca lineare per quanto riguardo i massimi locali. Il loro svantaggio consiste nel maggior numero di valutazioni richieste. Nel nostro contesto una valutazione impiega circa 3-10 secondi a seconda della complessita' del metodo di reperimento. Cio' permette di eseguire l'algoritmo di ottimizzazione in un tempo accettabile.

I parametri che abbiamo scelto di ottimizzare cambiano in base alla tecnica di reperimento (e quindi del laboratorio). Per il laboratorio 3 abbiamo scelto di ottimizzare $k_1, b$, ignoriamo $k_2$ in quando abbiamo visto che non ci sono termini ripetuti nelle query e quindi tale termine non influisce sul punteggio. Per il laboratorio 5 ottimizziamo $k_1, b, \alpha$. Per il laboratorio 7 ottimizziamo $k_1, b, \alpha, \beta, \gamma$. La funzione da massimizzare e' la Mean Average Precision.

Le figure \ref{fig:es_lab3}, \ref{fig:es_lab5} e \ref{fig:es_lab7} riportano l'andamento della MAP durante l'ottimizzazione della funzione di reperimento dei laboratori, rispettivamente lab 3, lab 5 e lab 7. La serie \textit{baseline} corrisponde alla MAP ottenuta con il laboratorio 3 prima dell'ottimizzazione. La serie \textit{lab3\_opt} corrisponde alla MAP ottenuta con il laboratorio 3 dopo l'ottimizzazione dei parametri.

\begin{figure}[htbp]
	\begin{center}
		\includegraphics[width=0.75\textwidth]{figures/es_lab3.png}
		\caption{MAP durante ottimizzazione laboratorio 3.}
		\label{fig:es_lab3}
	\end{center}
\end{figure}
\begin{figure}[htbp]
	\begin{center}
		\includegraphics[width=0.75\textwidth]{figures/es_lab5.png}
		\caption{MAP durante ottimizzazione laboratorio 5.}
		\label{fig:es_lab5}
	\end{center}
\end{figure}
\begin{figure}[htbp]
	\begin{center}
		\includegraphics[width=0.75\textwidth]{figures/es_lab5.png}
		\caption{MAP durante ottimizzazione laboratorio 7.}
		\label{fig:es_lab7}
	\end{center}
\end{figure}

Tabella \ref{tab:es} riporta i risultati dell'ottimizzazione.
\begin{table}[htdp]
\caption{Risultati ottimizzazione con ES per i vari laboratori.}
\begin{center}
\begin{tabular}{|c|c|c|}
\hline
laboratorio & parametri & MAP \\
 \hline
3 & $k_1 = 0.0253, b = 0.0221$ & $0.3354$ \\
5 & $k_1 = 0.0237, b = 0.0, \alpha=0.7832$ & $0.3366$ \\
7 & $k_1 = 0.0237, b = 0.0, \alpha=0.7832$ & $0.3366$ \\
\hline
\end{tabular}
\end{center}
\label{tab:es}
\end{table}

Come si puo' vedere grazie all'ES si e' ottenuto un notevole miglioramento in termini di MAP rispetto alla nostra prima versione. I risultati dell'ottimizzazione vengono discussi in Sezione \ref{sec:risult-sper}.

\subsection{Altri metodi}
tipo lucene
\label{sec:altri-metodi}

Se sono stati sviluppati altri metodi, descriverli qui.



\section{Risultati sperimentali}
\label{sec:risult-sper}

Per 2.3 facciamo 2 plot uno al variare di k uno al variare di b.

Per 2.4 farei un istogramma con le 3 map (no rf, rf esplicito, rf pseudo) e forse uno al variare di $N$ nello pseudo.

Per 2.5 solito pagerank a variare del nostro alpha.

Per 2.6 al variare di $N$.

Per 2.7 al variare di $N$?

Per 2.8 plot3d.

NDCG?

Un istogramma che mostra le map con iparametri ottimali per i vari metodi. Discussione su questi.

Efficienza?

%Questo paragrafo presenta e discute i risultati sperimentali. Si dovranno
%scegliere tre \textit{run} al massimo per ciascuno dei metodi illustrati nei
%paragrafi \ref{sec:metodi-di-reper}, \ref{sec:relevance-feedback},
%\ref{sec:pagerank}, \ref{sec:lsa} e \ref{sec:hits}.  
%
%Si dovranno confrontare le misure di efficacia (ad esempio, \textit{Mean Average
%  Precision}, MAP) mediante illustrazioni anche grafiche. Un'analisi della
%significativit\`a statistica delle differenze tra i valori di MAP sarebbe
%opportuna.
%
%Un confronto particolare dovr\`a essere fatto tra la \textit{baseline} del
%paragrafo \ref{sec:metodi-di-reper} e i metodi dei paragrafi successivi.
%
%La parte preziosa di questo paragrafo \`e la discussione dei risultati. Si
%dovr\`a dare un'interpretazione ragionata, chiara ed esaustiva delle ipotesi per
%cui sono state osservate o meno le differenze tra i valori di MAP. 



\section{Conclusioni}
\label{sec:conclusioni}
Nell'arco del corso di Sistemi Informativi sono state affrontate diverse tecniche per il reperimento. Durante le implementazioni abbiamo notato che non sempre metodi pi\`u avanzati portavano un miglioramento rispetto alla \textsc{baseline} e anche quando questo succedeva, la configurazione dei parametri era molto suscettibile e delicata. Inoltre la MAP ottenuta \`e relativamente bassa rispetto alle nostre aspettative iniziali. Siamo quindi arrivati a domandarci i motivi di questo comportamento. Analizzando il testo delle query abbiamo avuto modo di verificare come alcune di esse hanno bisogni informativi che riguardano informazioni che non sono presenti nel corpo di documenti indicizzati. Molte di esse chiedono infatti informazioni quali l'autore o l'anno di pubblicazione, altre invece esprimono regole di esclusione (eg. query 6: ``[...] We are not interested in the dynamics of arm motion.'') che non vengono considerate come tali dal modello che abbiamo utilizzato. Infine un altro aspetto che ha influenzato i risultati \`e il fatto che per molti documenti mancano gli abstract, e viene utilizzato solo il titolo. \`E difficile esprimere il vero contenuto informativo di un articolo utilizzando solamente le 2-10 parole del titolo. Riteniamo che questi fattori portino ad uno scarso miglioramento dell'efficacia del reperimento utilizzando tecniche avanzate, che potrebbero invece contribuire in maniera significativa su collezioni pi\`u grandi e complete. 

Considerando inoltre la variet\`a che c'\`e nella formulazione delle query (alcune contengono poche parole specifiche, altre invece testi pi\`u lunghi) ci rendiamo conto che a volte l'utilizzo dei semplici stem potrebbe non esprimere in modo adeguato il significato della query. Un esempio \`e la query 40: ``List all articles dealing with data types in the following languages: that are referenced frequently in papers on the above languages (e.g. catch any languages with interesting type structures that I might have missed)'' che corrisponde agli stem \textit{[ada, algol, alphard, articl, catch, clu, data, deal, el, frequent, interest, languag, list, miss, paper, pascal, referenc, russel, structur, typ]}.

Sarebbe interessante sperimentare delle tecniche di elaborazione delle query prima di proporle ad un sistema di IR; questo per evitare il fenomeno del \textit{topic drift} (interrogazioni fuori tema) presente in parecchie query pi\`u lunghe. Tecniche come la comprensione del linguaggio naturale risulterebbero utili nel comprendere il senso di una frase. Potrebbe essere utile l'implementazione di QE (Query Expansion), che permette di ridimensionare una query aggiungendo/togliendo parole e modificando i pesi dei termini. Una interrogazione ben formata avr\`a sicuramente pi\`u successo.

Abbiamo avuto modo di verificare il notevole miglioramento della MAP utilizzando \textsc{rf esplicito}, che non risente pesantemente di queste considerazioni, in quanto modifica l'ordinamento basandosi solamente su informazioni di rilevanza e occorrenze dei termini, e non utilizza aspetti pi\`u elaborati (e.g.: il grafo delle citazioni).

Un altro aspetto \`e stato gestire l'efficienza del reperimento. Sarebbe interessante verificare in che modo le tecniche realizzate si comportano con l'utilizzo di collezioni pi\`u grandi, e quali delle scelte effettuate per le implementazioni sulla collezione CACM si rivelano essere dei colli di bottiglia al crescere del corpo di documenti.

\bibliographystyle{abbrv}
\bibliography{references}

%In questo paragrafo si possono aggiungere delle osservazioni di carattere
%generale sugli esperimenti; ad esempio, si pu\`o concludere se un proprio metodo
%di reperimento o una variazione dei metodi pi\`u avanzati hanno portato a
%qualche miglioramento rispetto alla \textit{baseline}.

\end{document}
