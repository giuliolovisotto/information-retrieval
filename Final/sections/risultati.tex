\section{Risultati sperimentali}
\label{sec:risult-sper}

In questa sezione verranno discussi i risultati sperimentali dei vari laboratori. Se non specificato diversamente, la \textit{baseline} considerata e' quella del laboratorio 3 ottimizzato, i cui parametri sono riportati in Tabella \ref{tab:es}. 

2 DOMANDE:
Quale versione del lab devo considerare per il confronto. parto da stessi parametri e ottimizzo?
Perche solo 3 run?


Per 2.3 facciamo 2 plot uno al variare di k uno al variare di b.

Per 2.4 farei un istogramma con le 3 map (no rf, rf esplicito, rf pseudo) e forse uno al variare di $N$ nello pseudo.

Per 2.5 solito pagerank a variare del nostro alpha.

Per 2.6 al variare di $N$.

Per 2.7 al variare di $N$?

NDCG?

Un istogramma che mostra le map con iparametri ottimali per i vari metodi. Discussione su questi.

Efficienza?

%Questo paragrafo presenta e discute i risultati sperimentali. Si dovranno
%scegliere tre \textit{run} al massimo per ciascuno dei metodi illustrati nei
%paragrafi \ref{sec:metodi-di-reper}, \ref{sec:relevance-feedback},
%\ref{sec:pagerank}, \ref{sec:lsa} e \ref{sec:hits}.  
%
%Si dovranno confrontare le misure di efficacia (ad esempio, \textit{Mean Average
%  Precision}, MAP) mediante illustrazioni anche grafiche. Un'analisi della
%significativit\`a statistica delle differenze tra i valori di MAP sarebbe
%opportuna.
%
%Un confronto particolare dovr\`a essere fatto tra la \textit{baseline} del
%paragrafo \ref{sec:metodi-di-reper} e i metodi dei paragrafi successivi.
%
%La parte preziosa di questo paragrafo \`e la discussione dei risultati. Si
%dovr\`a dare un'interpretazione ragionata, chiara ed esaustiva delle ipotesi per
%cui sono state osservate o meno le differenze tra i valori di MAP. 

