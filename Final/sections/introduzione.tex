\section{Introduzione}
\label{sec:introduzione}

% Introdurre brevemente lo scopo del corso.
Il corso di Sistemi Informativi ha come obiettivo l'insegnamento di nozioni basilari di Information Retrieval e Motori di Ricerca. Gli argomenti trattati a lezione ed i concetti applicati durante i vari laboratori sono stati utili a comprendere il funzionamento di un reale sistema di Information Retrieval.

I progetti svolti nelle lezioni di laboratorio prevedono l'utilizzo della CACM collection, che \`e una collezione di articoli scientifici tratti dalla rivista \textit{Communications of the ACM}~\cite{ACMCollection}. La collezione contiene i titoli, gli abstract, e altre informazioni quali: autori, luogo della conferenza, e la lista di citazioni verso altri documenti della collezione. Inoltre \`e disponibile un set di query e relativi giudizi di rilevanza. Sulla collezione CACM la ricerca si \`e concentrata sull'utilizzo dei soli titoli e abstract per il reperimento, quindi verra' utilizzato tale approccio. Grazie all'insieme delle citazioni \`e possibile sperimentare nella funzione di reperimento l'uso di \textsc{pagerank} e \textsc{hits} che utilizzano il grafo costruito a partire delle citazioni per valutare la rilevanza di un documento. Inoltre \`e stato implementato un algoritmo di Evolution Strategy per l'ottimizzazione degli algoritmi di reperimento trattati. L'esposizione del presente documento si svolge secondo criteri di chiarezza e sintesi. Per tutti i dettagli che non vengono descritti si rimanda agli autori e ai testi di riferimento~\cite{manning2008introduction,melucci2013information,croft2010search}.
\subsection{Organizzazione}
Il resto del documento \`e organizzato come segue. In Sezione \ref{sec:metodologia} vengono descritti i metodi sviluppati per ciascun laboratorio. In Sezione \ref{sec:risult-sper} vengono discussi i risultati ottenuti per i vari laboratori, e vengono descritte alcune considerazioni sull'efficienza dei metodi utilizzati. Infine, in Sezione \ref{sec:conclusioni} conclude il documento con alcune conclusioni generali.

%Quando si scrive l'introduzione, si deve tenere presente che il lettore atteso
%sar\`a a conoscenza degli elementi di base. Si dovranno quindi introdurre i
%concetti non normalmente trattati in un corso di base in Information Retrieval e
%che sono invece utilizzati nel proprio documento.
%
%Il resto del documento sar\`a scritto con i criteri seguenti:
%\begin{itemize}
%\item esaustivit\`a
%\item precisione
%\item chiarezza
%\item correttezza
%\item sintesi
%\end{itemize}
