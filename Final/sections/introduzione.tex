\section{Introduzione}
\label{sec:introduzione}

http://www.guit.sssup.it/latex/suggerimenti.php

Introdurre brevemente lo scopo del corso. Che collezione usiamo (http://ir.dcs.gla.ac.uk/resources/test\_collections/cacm/). No stop words, no stemming (gia fatto). 
Abbiamo a disposiz il grafo delle citazioni quindi useremo \textsc{pagerank} e Hits (link analysis), spiegare brevemente cosa sono questi. Spiegare LSA. Spiegare ES. Related works?  citt\`e
\subsection{Organizzazione}
In sezione .. spiegeremo .., in sezione .. spiegheremo .. etc.

%Quando si scrive l'introduzione, si deve tenere presente che il lettore atteso
%sar\`a a conoscenza degli elementi di base. Si dovranno quindi introdurre i
%concetti non normalmente trattati in un corso di base in Information Retrieval e
%che sono invece utilizzati nel proprio documento.
%
%Il resto del documento sar\`a scritto con i criteri seguenti:
%\begin{itemize}
%\item esaustivit\`a
%\item precisione
%\item chiarezza
%\item correttezza
%\item sintesi
%\end{itemize}
