\section{Introduzione}
\label{sec:introduzione}

% Introdurre brevemente lo scopo del corso.
Il corso di Sistemi Informativi ha come obiettivo l'insegnamento di nozioni basilari di Information Retrieval e Motori di Ricerca. Gli argomenti trattati a lezione ed i concetti applicati durante i vari laboratori sono stati utili a comprendere il funzionamento di un reale sistema di Information Retrieval.

Il progetto svolto nelle lezioni di laboratorio prevede l'utilizzo di una collezione di dati fornita dal docente; la collezione utilizzata \'e una raccolta di articoli scientifici tratti dalla rivista \textit{Communications of the ACM} (http://cacm.acm.org/)
La collezione di documenti \'e stata fornita come un unico file xml dove ogni nodo di documento contiene un identificativo, un titolo e altre informazioni come abstract, autori, luogo, riferimenti e citazioni ad altri documenti della collezione.
Oltre al file contenente i documenti sono stati forniti file contenenti le parole chiave sotto forma di stem, un insieme di query e l'insieme delle citazioni tra i documenti.

%Durante le lezioni di laboratorio gli studenti, organizzati in piccoli gruppi, hanno avuto modo di implementare diverse tecniche nel proprio progetto che ....

\vspace{30px}
Che collezione usiamo (http://ir.dcs.gla.ac.uk/resources/test\_collections/cacm/). No stop words, no stemming (gia fatto). 
Abbiamo a disposiz il grafo delle citazioni quindi useremo \textsc{pagerank} e Hits (link analysis), spiegare brevemente cosa sono questi. Spiegare LSA. Spiegare ES. Related works?  citt\`e
\subsection{Organizzazione}
In sezione .. spiegeremo .., in sezione .. spiegheremo .. etc.

%Quando si scrive l'introduzione, si deve tenere presente che il lettore atteso
%sar\`a a conoscenza degli elementi di base. Si dovranno quindi introdurre i
%concetti non normalmente trattati in un corso di base in Information Retrieval e
%che sono invece utilizzati nel proprio documento.
%
%Il resto del documento sar\`a scritto con i criteri seguenti:
%\begin{itemize}
%\item esaustivit\`a
%\item precisione
%\item chiarezza
%\item correttezza
%\item sintesi
%\end{itemize}
