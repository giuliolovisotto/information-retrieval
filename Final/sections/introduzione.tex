\section{Introduzione}
\label{sec:introduzione}

% Introdurre brevemente lo scopo del corso.
Il corso di Sistemi Informativi ha come obiettivo l'insegnamento di nozioni basilari di Information Retrieval e Motori di Ricerca. Gli argomenti trattati a lezione ed i concetti applicati durante i vari laboratori sono stati utili a comprendere il funzionamento di un reale sistema di Information Retrieval.

Il progetto svolto nelle lezioni di laboratorio prevede l'utilizzo di una collezione di dati fornita dal docente; la collezione utilizzata (disponibile anche online) \'e una raccolta di articoli scientifici tratti dalla rivista \textit{Communications of the ACM} \cite{ACMCollection} %TODO aggiungere il cite ACMCollection nella bibliografia (url: http://ir.dcs.gla.ac.uk/resources/test_collections/cacm/)
La collezione di documenti \'e stata fornita come un unico file xml dove ogni nodo di documento contiene un identificativo, un titolo e altre informazioni come abstract, autori, luogo, riferimenti e citazioni ad altri documenti della collezione. Per il reperimento di documenti utili ad una query vengono considerati titolo ed abstract di ogni documento.
Oltre al file contenente i documenti sono stati forniti file contenenti le parole chiave sotto forma di stem, un insieme di query e l'insieme delle citazioni tra i documenti.
Grazie all'insieme delle citazioni \'e stato possibile sperimentare nella funzione di reperimento l'uso di \textsc{pagerank} e \textsc{HITS} che considerano il grafo delle citazioni per valutare la rilevanza di un documento.
\vspace*{30px}

\textit{Manca ancora: Introduzione su Evolutionary Strategy? Related works?}
\subsection{Organizzazione}
Il resto del documento \'e organizzato come segue.\\
Nella seconda sezione sono illustrati i metodi sviluppati per ciascun laboratorio. La sottosezione \ref{sec:approccio} descrive l'approccio generale del gruppo durante il lavoro svolto.
Nella sezione 3 sono discussi i risultati ottenuti dai laboratori, con qualche considerazione sull'efficienza dei metodi utilizzati (sottosezione \ref{sec:efficienza})
Infine, la sezione 4 conclude il documento con alcune considerazioni personali del gruppo.

%Quando si scrive l'introduzione, si deve tenere presente che il lettore atteso
%sar\`a a conoscenza degli elementi di base. Si dovranno quindi introdurre i
%concetti non normalmente trattati in un corso di base in Information Retrieval e
%che sono invece utilizzati nel proprio documento.
%
%Il resto del documento sar\`a scritto con i criteri seguenti:
%\begin{itemize}
%\item esaustivit\`a
%\item precisione
%\item chiarezza
%\item correttezza
%\item sintesi
%\end{itemize}
