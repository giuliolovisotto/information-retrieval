
\title{Sistemi Informativi \\ Laboratorio 5}
\author{
        Catalin Copil
            \and
        Mattia de Stefani
            \and
        Giulio Lovisotto
}
\date{\today}

\documentclass[12pt]{article}
\usepackage{algorithmicx}
\usepackage{algpseudocode}
\usepackage{graphicx}
\usepackage{geometry}

\addtolength{\topmargin}{-.5in}
\begin{document}
\maketitle

\section{Descrizione}
Computeremo il \textsc{pagerank} con la libreria \texttt{networkx} a tempo di indexing e salveremo i risultati su un file ($id \rightarrow pagerank$). La nostra funzione di reperimento combinera' gli score di BM25 ($rk$) con pagerank ($pr$) nel seguente modo:

\[ score =  \alpha \cdot \ln (rk+1) + (1-\alpha) \cdot pr,\]

dove $\alpha$ e' un parametro tra 0 e 1 che determina l'importanza del ranking (primo termine) e del pagerank (secondo termine). E' stato introdotto tale parametro per evitare che il valore del pagerank influisse troppo sul punteggio finale. Sperimentalmente proveremo diversi valori per $\alpha$ e sceglieremo quello che garantisce la massima $map$. Abbiamo scelto di usare il logaritmo naturale per ridimensionare l'ordine di grandezza dello score di BM25 (i cui score valgono anche piu di 10), e di sommare 1 per evitare valori negativi del logaritmo.

\bibliographystyle{abbrv}
\bibliography{main}

\end{document}