
\title{Sistemi Informativi \\ Laboratorio 7}
\author{
        Catalin Copil
            \and
        Mattia de Stefani
            \and
        Giulio Lovisotto
}
\date{\today}

\documentclass[12pt]{article}
\usepackage{algorithmicx}
\usepackage{algpseudocode}
\usepackage{graphicx}
\usepackage{geometry}

\addtolength{\topmargin}{-.5in}
\begin{document}
\maketitle

\section{Descrizione}
La nostra funzione di reperimento utilizzera' i primi $N$ documenti prodotti da BM25 per costruire il grafo delle citazioni $R_q$, tale grafo verra' espanso per ottenere un grafo allargato $B_q$, e su quest'ultimo verra' calcolato \textsc{hits}. La funzione combinera' poi gli score di BM25 ($sc$) con i punteggi di authority ($auth$) e hubbiness ($hub$) per i primi $N$ documenti, e li riordinera' per $hits_{score}$ decrescente:

\[ hits_{score} =  \alpha \cdot sc + \beta \cdot auth + \gamma \cdot hub,\]

$\alpha$ varia in [0, 1],  mentre $\beta, \gamma$ sono parametri che possono variare tra [-1, 1], in quanto vogliamo cogliere possibili influenze negative sul di tali valori. Nella precedente funzione i valori di $sc, auth, hub$ vengono normalizzati in [0, 1] prima del calcolo. 

 Vogliamo provare diverse combinazioni di tali valori per ottenere la massima precisione.

\bibliographystyle{abbrv}
\bibliography{main}

\end{document}